\chapter*{Введение}
\addcontentsline{toc}{chapter}{Введение}

В настоящее время большую актуальность имеют системы мониторинга состояния загруженности операционной системы. Особое внимание уделяется операционным система с ядром Linux \cite{linux}.

В современном мире на большой части серверов используется именно такие операционные системы. На таких серверах размещаются специальные хранилища с пользовательскими данными, Web-приложения и так далее. За любым из таких серверов нужно наблюдать: в любой момент могут возникнуть сбои, что может привести к потери данных пользователя или недоступности какого-либо ресурса, что в своё время может привести к денежным потерям.

Для обнаружения и предотвращения сбоев необходимо иметь хорошую систему мониторинга, которая будет анализировать работу операционной системы. Данный курсовой проект посвящен исследованию структур ядра, хранящим информацию о процессах в системе и памяти, и способам перехвата системных вызовов ядра с их последующим логированием.

Целью данной курсовой работы является разработка загружаемого модуля ядра, предоставляющего информацию о загруженности системы: количество системных вызовов за выбранный промежуток времени, количество выделенной памяти, статистика по процессам и в каких состояниях они находятся.

Для достижения поставленной цели необходимо выполнить следующие задачи:

\begin{itemize}
	\item изучить структуры и функции ядра, которые предоставляют информацию о процессах и памяти;
	\item проанализировать существующие подходы к перехвату системных вызовов и выбрать наиболее подходящий;
	\item реализовать загружаемый модуль ядра.
\end{itemize}
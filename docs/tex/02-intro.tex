\chapter*{Введение}
\addcontentsline{toc}{chapter}{Введение}

В настоящее время большую актуальность имеют системы, предоставляющие информацию о ресурсах операционной системы и частоте системных вызовов. Предоставив такую информацию, пользователь сможет проанализировать состояние системы и нагрузку на неё. Особое внимание уделяется операционным системам с ядром Linux \cite{linux}. Ядро Linux возможно изучать благодаря тому что оно имеет открытый исходный код. 

Данная работа посвящена исследованию структур ядра, хранящим информацию о процессах в системе и памяти, способам перехвата системных вызовов ядра с их последующим логированием.

Целью данной курсовой работы является разработка загружаемого модуля ядра, предоставляющего информацию о количестве системных вызовов и выделенной памяти за выбранный промежуток времени и информацию о состоянии всех процессов в системе в текущий момент.

%Для достижения поставленной цели необходимо выполнить следующие задачи:

%\begin{itemize}
	%\item изучить структуры и функции ядра, которые предоставляют информацию о процессах и памяти;
	%\item проанализировать существующие подходы к перехвату системных вызовов и выбрать наиболее подходящий;
	%\item реализовать загружаемый модуль ядра.
%\end{itemize}